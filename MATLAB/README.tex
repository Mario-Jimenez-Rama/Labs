\documentclass{article}
\usepackage{graphicx} % Required for inserting images

\title{Adeline}
\author{Mario Jiménez Rama}
\date{October 2024}

\begin{document}

\maketitle

\section{Content}
This is my numerical simulation using the method of finite differences for solving the homogenous laplacian with different boundary conditions. The code in MATLAB is 100\% original but influenced by the books of  \textit{Numerical Mathematics} (Springer-Verlag Berlin Heidelberg (2007) - Alfio Quarteroni, Riccardo Sacco, Fausto Saleri) and \textit{computational partial differential equations using matlab} ( Jichun Li, Yi-Tung Chen). 
My main goal was to use an efficient code, that is why there´s no use of loops, instead,  the kronicher matrix is employ. In this way, not only the code is more efficient but also, it is easier to make modifications for new conditions (boundary conditions, domain, inhomogenous terms). 

Yoy will find the next files: 
\begin{enumerate}
    \item README
    \item Laplace`s codes
    \item Explanation
\end{enumerate}

\section{Laplace Problems}
\subsection{Simulation 1}
\[ \left\{
  \begin{array}{lr}
    \Delta u = 0 \hspace{131pt} 0<x<1,0<y<1\\
    u(0,y) = 0 , u(1,y) = y^{2}  -101y + 50 \hspace{7pt} 0<y<1 \\
    u(x,0) = 50x , u(x,1) = -50x \hspace{37pt} 0<x<1 \\
  \end{array}
\right.
\]
\subsection{Simulation 2}
\[ \left\{
  \begin{array}{lr}
    \Delta u = 0 \hspace{130pt} 0<x<2,0<y<1\\
    u(0,y) = 0 , u(2,y) = 100 \hspace{55pt} 0<y<1 \\
    u_y(x,0) = -20 , u_y(x,1) = 40 \hspace{37pt} 0<x<2 \\
  \end{array}
\right.
\]
\subsection{Simulation 3}
\[ \left\{
  \begin{array}{lr}
    \Delta u = 0 \hspace{128pt} 0<x<2,0<y<1\\
    u(0,y) = -50 , u(2,y) = 100 \hspace{40pt} 0<y<1 \\
    u(x,0) = -50 , u(x,1) = -50 \hspace{36pt} 0<x<2 \\
  \end{array}
\right.
\]

Thank you very much, I hope you'll find it usefull ;)

\end{document}